\documentclass[12pt, letterpaper]{article}
\usepackage[utf8]{inputenc}
\usepackage{geometry}
\usepackage[english]{babel}
% \setlength{\parindent}{4em}
\setlength{\parskip}{1em}
\renewcommand{\baselinestretch}{1.5}

\geometry{a4paper,left=1.25in, right= 1.25in, top=1in, bottom=1in}
% \usepackage [backend=biber, style=ieee]{biblatex}
% \addbibresource{references.bib}

\usepackage[square,numbers]{natbib}
\bibliographystyle{abbrvnat}
\setcitestyle{authoryear,open={(},close={)}}


\title {\textbf{\large Literature Review: Home Healthcare Routing Problem}}
\author {Sudhan Bhattarai}
\date {October 2020}
\begin{document}
\maketitle

A home healthcare provider is a type of company destined to provide a periodic medical service to the patients at their own houses. Consequently, a challenge arises to build an efficient visit plan for the human resources available. The challenge is commonly known as Home Healthcare Routing Problem (HHCRP). The HHCRP is similar to the traditional Vehicle Routing Problem (VRP) in which a fleet of vehicles available at a depot goes out to pick or deliver goods to the customers located in different places and finally returns back to the depot. Similarly, in a HHCRP, a fleet of nurses available at the home care center goes out on a tour to serve the patients at their home and finally comes back to the center. Moreover, the problem can be made more realistic by using the standards for nurses' start time, end time, service duration for each patient, qualification of the nurses and more of the patients' demands. My research is about formulating an efficient routing plan for the nurses' such that the total visits are maximized. There have been many remarkable researches in the past to build the realistic yet effectual routing plans which are briefly described in this review.

\end{document}
